\documentclass[12pt,article,oneside]{memoir}
\usepackage{latexsym}
\usepackage[activeacute,spanish]{babel}
\usepackage[colorlinks=true, urlcolor=blue, linkcolor=red]{hyperref}

\setcounter{secnumdepth}{5} % Note that part is -1 level !
\setcounter{tocdepth}{5}

\title{Desarrollador Fullstack Senior\\T\'opicos para entrevista}
\author{Jorge Rivera\\jorge.rivera@ssde.com.mx}
\date{July 2024}

\begin{document}
\maketitle
\newpage
\tableofcontents*
\newpage

\part{  Generales}
	\chapter{Patrones de dise\~no}
	Un patr\'on de dise\~no es una soluci\'on general repetible a un problema que ocurre comunmente. No es un dise\~no 
	terminado que pueda convertirse directamente a c\'odigo. Es una descripci\'on o plantilla de c\'omo resolver un problema 
	que puede ser utilizado en muchas situaciones diferentes.\\
	
	\textbf{Usos de patrones de dise\~no}\\
	
	Los patrones de dise\~no pueden acelerar el proceso de desarrollo a trav\'es de paradigmas de desarrollo probados y 
	comprobados. El dise\~no de software efectivo requiere considerar problemas que pueden no ser visibles  sino hasta 
	tarde en la implementaci\'on. Reusar patrones de dise\~no ayuda a prevenir problemas sutiles que pueden generar problemas
	mayores y mejora la legibilidad del c\'odigo para desarrolladores y arquitectos familiarizados con los patrones.\\
	
	Los patrones de dise\~no comunes pueden ser mejorados iterativamente haciendolos m\'as robustosque dise\~nnos a medida.\\
	
	Los patrones pueden dividirse en 3 tipos
	
	\begin{itemize}
		\item Patrones creacionistas,
		\item Patrones estructurales y
		\item Patrones de comportamiento
	\end{itemize}
	\newpage
	
		\section{Patrones creacionistas}
		
		\'Estos patrones se tratan de instanciaci\'on de clases . Pueden dividirse a su vez en patrones de creaci\'on de 
		clases y patrones de creaci\'on de objetos. Mientras que los primeros usan la herencia efectivamente en el proceso 
		de instanciaci\'on, los segundos usan la delegaci\'on efectivamente para completar su trabajo.\\
		
		Algunos ejemplos de \'este tipo de patr\'on son:
		
		\begin{itemize}
			\item \textbf{Fabrica abstracta (Abstract Factory):}\\
			Crea una instancia de varias familias de clase. El cliente 
			de software	usa un interfaz gen\'erica de la fabrica para crear objetos concretos que son parte de la familia. 
			El cliente no sabe que objetos concretos recibe de cada una de \'estas f\'abricas internas porque usa s\'olo las 
			interfaces gen\'ericas de sus productos
			
			\item \textbf{Constructor (Builder):}\\
			Separa la representaci\'on de un objeto de su implementaci\'on. \'Este 
			patr\'on a diferencia de tros patrones creacionistas no requiere que se tengan interfaces comunes. \'Esto hace 
			posible crear diferentes productos usango el mismo proceso de construcci\'on. \'Este patr\'on puede reconocerse 
			en una clase que tiene un solo metodo para crearla y varios metodos para configurar el objeto resultante. Lo objetos 
			de \'este patr\'on comunmente soportan el encadenado. Ex:\\
			\begin{verbatim}
			 someBuilder.setValueA(1).setValueB(2).create();
			\end{verbatim}
			
			\item \textbf{M\'etodo de F\'abrica (Factory Method):}\\
			\'Este patr\'on de dise\~no provee una forma de crear objetos pero permite a las subclases alterar el tipo de 
			objeto que ser\'a creado. Si tenemos un m\'etodo de creaci\'on en la clase base subclases que la extienden, 
			pudieramos estar hablando de \'este patr\'on. Ex:
			\begin{verbatim}
			abstract class Department {
				public abstract function createEmployee($id);
				
				public function fire($id) {
        			$employee = $this->createEmployee($id);
        			$employee->paySalary();
        			$employee->dismiss();
    			}
			}
			
			class ITDepartment extends Department {
				public function createEmployee($id) {
					return new Programmer();
				}
			}
			
			class AccountingDepartment extends Department {
				public function createEmployee($id) {
					return new Accountant();
				}
			}
			\end{verbatim}
			
			\item \textbf{Alberca de objetos (Objects pool):}\\
			Provee una alberca de objetos pre-inicializados que pueden ser reutilizados en lugar de crearlos y destruirlos 
			bajo demanda. \'Esto mejora el desempe\~no y reduce el sobre-trabajo de crear y destruir objetos
			
			\item \textbf{Prototipo (Prototype):}\\
			Las clases prototipo deben tener una interfaz com\'un que haga posiblecopiar objetos a\'un si sus clases concretas 
			son desconocidas. Los objetos prototipo puedenn producir copias completas debido a que los objetos de la misma 
			clase pueden acceder a las propiedades privadas una de la otra.
			
			\item \textbf{Singleton:}\\
			En \'este patr\'on de dise\~no s\'olo puede existir una instancia de cada clase.
			\newpage
			
		\end{itemize}
		\newpage
	
		\section{Patrones estructurales}
		\'Estos patrones se tratan de composici\'on de clases y objetos. \'Estos patrones de creaci\'on de clases usan la 
		herencia para componer interfaces. Se definen formas de acomodar objetos para obtener nueva funcionalidad.\\
		
		Algunos ejemplos de \'este tipo de patr\'on son:
		\begin{itemize}
			\item \textbf{Adaptador (Adapter):} Empata interfaces con distintas clases. Permite que objetos con distintas 
			interfaces colaborar.
			
			\item \textbf{Puente (Bridge):} \'Este es un patr\'on que nos permite dividir un clase grande o un conjunto de 
			clases relacionadas estrechamente en 2 jerarqu\'ias separadas \textit{abstracci\'on} e \textit{implementaci\'on}, 
			las cuales pueden desarrollarse independientes una de otra.\\
			
			En \'este patr\'on los t\'erminos abstracci\'on e implementaci\'on no son los mismos que utilizamos en nuestros
			lenguajes de programaci\'on, \'estos se definen a continuaci\'on:
			\begin{itemize}
				\item[+] \textbf{Abstracci\'on:} Es la capa de control de m\'as alto nivel. \'Esta capa no se supone que haga 
				ning\'un tipo de trabajo. \'Esta debe solo delegar el trabajo a la capa de implementaci\'on (plataforma).
				
				\item[+] \textbf{Implementaci\'on:} \'Esta se refiere a la capa donde se procesan nuestras peticiones de la 
				capa anterior, por ejemplo una API o un servicio.
			\end{itemize}
			
			\item \textbf{Compuestos (Composed):} \'Este patr\'on nos permite componer objetos en estructura de \'arbol y 
			luego trabajar con \'estas estructuras como si fueran objetos individuales. 
			M\'as detalles \href{https://refactoring.guru/design-patterns/composite}{aqu\'i}
			
			\item \textbf{Decorador (Decorator):} Permite adjuntar nuevos comportamientos a objetos al ponerlos dentro de 
			objetos envoltorio que contienen dichos comportamientos.
			M\'as detalles \href{https://refactoring.guru/design-patterns/decorator}{aqu\'i}
			
			\item \textbf{Facade:} Provee una interfaz simplificada a una librer\'ia, framework o alg\'un otro conjunto de 
			clases complejas. M\'as detalles \href{https://refactoring.guru/design-patterns/facade}{aqu\'i}\\ \\
			
			\item \textbf{Peso mosca (flyweight):} Permite acomodar m\'as objetos en la cantidad disponible de RAM 
			compartiendo partes de su estado entre m\'ultiples objetos en lugar de mantener todos los datos en cada 
			objeto. M\'as detalles \href{https://refactoring.guru/design-patterns/flyweight}{aqu\'i}
			
			\item \textbf{Datos de clase privada (Private data class):} Restringe el acceso al estado interno de un objeto
			mejorando la seguridad y reduciendo los riesgos de corrupci\'on de datos por medio de m\'etodos de acceso 
			controlados
			
			\item \textbf{Proxy:} Provee un sustituto o ancla para otro objeto. Un proxy controla el acceso al objeto 
			original permitiendonos realizar acciones ya sea antes o despu\'es de que el requerimiento llega al objeto 
			original. M\'as detalles \href{https://refactoring.guru/design-patterns/proxy}{aqu\'i}
		\end{itemize}
		\newpage
	
		\section{Patrones de comportamiento}
		\'Estos patrones se tratan de la comunicaci\'on de clases de objetos. \'Estos patrones son aquellos que est\'an
		m\'as espec\'ificamente preocupados por la comunicaci\'on entre objetos.\\
		
		Algunos ejemplos de \'estos patrones son:
		
		\begin{itemize}
			\item \textbf{Cadena de responsabilidad (Cahin of Responsibility):} Permite pasar requerimientos a trav\'es de 
			una cadena de manejadores. Al recibir el requerimiento cada manejador decide ya sea procesarlo o pasarlo al 
			siguiente manejador en la cadena. 
			M\'as detalles \href{https://refactoring.guru/design-patterns/chain-of-responsibility}{aqu\'i}
		\end{itemize}
		\newpage
		
	\chapter{AWS}
	\newpage
	
	\chapter{Docker/Kubernates}
	
\part{  Desarrollo Back End}
	\chapter{Java Core}	
	Java fue desarrollado originalmente por James Gosling para Sun Microsystems en Mayo de 1995 como un componente base de la 
	plaaforma de Java de Sun. Las librerias, compiladores y m\'aquinas virtuales originales fueron lanzadas bajo licencias 
	propietarias de Sun. A partir de 2007 en cumplimiento con las nirmas de Java Community Process, Sun cambi\'o su licencia 
	de la mayr\'ia de sus tecnolog\'ias Java a la licencia GPL-2.0-only. Oracle tiene su propia implementaci\'on de m\'aquina 
	virtual, sin embargo la versi\'on oficial es la de OpenJDK la cual es gratuita, de c\'odigo abierto y la m\'as utilizada 
	por los desarrolladores y es la versi\'on defacto utilizada por casi todas las distribuciones de Linux.\\
	
	A continuaci\'on se detallan varias particularidades de \'este lenguaje.
	
		\section{Tipos de dato}
		Java no es un LPOO completo porque tenemos tipos de datos primitivos, ie; int, long, boolean, double as\'i como sus 
		clases contenedores Integer, Long, Double, Boolean. \\
		
		En un LPOO todos los elementos son objetos, no hay tipos primitivos.
		
		\section{Modificadores de acceso}
		Todos los objetos en Java pueden restringirse usando los modificadores public, private protected y default, \'este 
		\'ultimo no es necesario utilizarlo, al no utilizar ning\'un modificador se toma \'este por defecto, a continuaci\'on 
		se definen los alcances de los diferentes modificadores.
		
			\begin{itemize}
				\item \textbf{default:} El alcance de \'este modificador es a nivel de paquete, esto es, cualquier objeto dentro 
				del mismo paquete puede acceder a la propiedad o m\'etodo.
				
				\item \textbf{private:} \'Este modificadores el m\'as restrictivo, permite acceso solo a los elementos de la 
				clase a la cual pertenecen.
				
				\item \textbf{protected:} Permiten acceso a elementos del mismo paquete o sus subclases inclusive si \'estas no 
				est\'an dentro del mismo paquete.
				
				\item \textbf{public:} \'Este es el m\'as permisivo de todos, permite acceso desde cualquier clase o metodo
				dentro de la aplicaci\'on
			\end{itemize}
			
		Los modificadores son importantes porque nos permiten:
		
			\begin{itemize}
				\item \textbf{Encapsulamiento:} Nos permiten encapsular codigo en clases y exponer solo las partes del 
				c\'odigo que requiere acceso. \'Esto reduce la dependencia entre clases.

				\item \textbf{Previene mal uso:} Previene mal uso de metodos y propiedades en formas que no se tienen 
				planificadas.
				
				\item \textbf{Seguridad:} Restringir el acceso a datos y metodos sensibles mejora la seguridad porque 
				esconde detalles de implementaci\'on a potenciales atacantes.
				
				\item \textbf{Refactorizaci\'on:} El c\'odigo que utiliza modificadores apropiados es m\'as f\'acil de 
				simplificar porque reducir la visibilidad no rompe otros c\'odigo.
				
				\item \textbf{Legibilidad:} Hacen el c\'odigo m\'as facil de leer porque expl\'icitamente se asume 
				quienes pueden acceder a los componentes.
				
				\item \textbf{Reuso:} Las clases que hacen buen uso de los modificadores ser\'an m\'as f\'aciles de 
				reutilizar en otros proyectos sin modificaciones extensas.
				
				\item \textbf{Control de Interfaz:} Permiten definir interfaces publicas estables para clases mientras 
				se mantiene la implementaci\'on privada
			\end{itemize}
			
			Para mayor informaci\'on referente a modificadores de acceso y su implementaci\'on visite la 
			\href{https://www.simplilearn.com/tutorials/java-tutorial/access-modifiers#what_are_access_modifiers_in_java}{liga}.			
		\section{Gen\'ericos}
		Fueron introducidos en Java 1.5, se buscaba reducir los errores y adicionar una capa extra de abstracci\'on 
		a los tipos.\\
		
		Imaginemos que tenemos el c\'odigo siguiente:
			\begin{verbatim}
				List list = new LinkedList();
				list.add(new Integer(1)); 
				Integer i = list.iterator().next();
			\end{verbatim}
		
		Sorprendentemente el compilador nos mandar\'a un error en la \'ultima l\'inea porque no sabe el tipo de dato que 
		se regresa. El compilador necesita un cast expl\'icito:
			\begin{verbatim}
				Integer i = (Integer) list.iterator.next();
			\end{verbatim}
		
		Nada nos garantiza que el tipo de dato devuelto por la lista es un $Integer$, la lista que definimos puede contener 
		objetos de cualquier tipo. Nosotros solo sabemos que estamos recibiendo una lista al inspeccionar el contexto. Cuando 
		vemos los tipos, solo podemos garantizar que es un objeto y por lo tanto se necesita de un cast expl\'icito para 
		asegurar que el tipo es el adecuado.\\
		
		Hacer cast es engorroso - sabemos que el tipo de dato en la lista en un $Integer$. El cast tambi\'en nos infla 
		el c\'odigo y puede generar errores en tiempo de ejecuci\'on relacionados a tipos de datos si un programador 
		comete un error con la conversi\'on de tipos.\\
		
		Lo m\'as sencillo es que los programadores expresen sus intenciones de utilizar un tipo espec\'ifico de dato y que el 
		compilador se asegure de la correctez de dichos tipos. \'Esta es la idea detr\'as de los gen\'ericos.\\
		
		Modifiquemos el c\'odigo para incluir los genericos:
			\begin{verbatim}
			List<Integer> list = new LinkedList<>();
			\end{verbatim}
		Al agregar el operador diamante $<>$ conteniendo el tipo de dato, especializamos la lista para solo utilizar el tipo 
		$Integer$. En otras palabras, especificamos el tipo de dato que la lista contiene. Em compilador entonces puede 
		forzar el tipo al tiempo de compilar.\\
		
		En programas peque\~nos puede verse trivial pero para programas m\'as grandes \'esto puede agregar robustez y hacer 
		el c\'odigo f\'acil de leer.
		
		\section{Map, HashMap, LinkedHashMap, TreeMap, ConcurrentHashMap}
		
			\begin{itemize}
				\item \textbf{Map:} Es una interfaz con una correspondencia clave-valor

				\item \textbf{HashMap:} Es un Map que utiliza una hash table para su implementación. Permite nulos en claves 
				o valores
				
				\item \textbf{HashTable:} Es una versión sincronizada de HashMap. No permite nulos en claves o valores.
				
				\item \textbf{TreeMap:} Usa un árbol para implementar un Map. Ordena los elementos de acuerdo a un iterador, 
				si no se especifica uno se oprdenan naturalmente en orden ascendente
				
				\item \textbf{ConcurrentHashMap:} Permite a varios hilos que lo accedan al mismo tiempo y de forma segura
				
				\item \textbf{LinkedHashMap:} Conserva el orden de iteración de los objetos que fueron insertados (otros no 
				proporcionan un orden de iteración fijo)
			\end{itemize}
			
			\'Esta misma diferencuaci\'on puede hacerse con las colleciones List y Set. 
		
	\chapter{Cambios entre versiones}
	A trav\'es de los a\~nos \'este lenguaje ha recibido una considerable cantidad de mejoras, desde su introducci\'on en 
	el a\~no 1995 hasta la fecha.\\
	
	Java gan\'o mucha popularidad desde su lanzamiento y ha sido popular desde entonces. En 2022 era el 3er lenguaje 
	m\'as popular seg\'un Github. Aunque es vastamente utilizado, ha habido un declive en su uso en a\~nos recientes.\\
	
	Hasta Marzo de 2024, Java 22 es la \'ultima version. Java 8, 11 17 y 21 son las anteriores versiones LTS que todav\'ia 
	tienen soporte.\\
	
	
		\section{Cambios Java 8}
		\'Esta versi\'on fue la m\'as esperada actualizaci\'on de Java debido a que en toda la historia de \'este lenguaje 
		no hab\'ian sido liberadas tantos cambios significativos. \'Esta liberaci\'on se di\'o el 18 de Marzo de 2014. \'Esta 
		nueva versi\'on vino acompa\~nada de soporte para programaci\'on funcional, nuevas APIs, un nuevo motor de JavaScript 
		y otros cambios que ser\'an detallados a continuaci\'on \\
			
			\subsection{Expresiones lambda}
			\'Estas permiten una representaci\'on concisa de funciones an\'onimas, habilitando paradigmas de programaci\'on 
			funcional en Java.\\
			
			Una expresi\'on lambda consiste en una lista de par\'ametros, un token de flecha $\rightarrow$ y un cuerpo. Son 
			particularmente  \'utiles para implementar interfaces de un solo m\'etodo (interfaces funcionales) y para pasar
			comportamiento como par\'ametro a un m\'etodo.
			
			\'Estas expresiones ayudan a reducir codigo com\'un y para hacer el c\'odigo m\'as legible y mantenible.
			
			\subsection{Interfaces funcionales}
			Se introduce la anotaci\'on $@FunctionalInterface$ para definir las interfaces con un solo metodo abstracto. \'Estas 
			interfaces facilitan el uso de expresiones lambda y referencias de metodo, habilitando los patrones de programaci\'on 
			funcional.\\
		
			\subsection{Introducci\'on y mejora de APIs}
				\subsubsection{Introducci\'on de API Stream}
				\subsubsection{Introducci\'on de API Date/Time}
				\subsubsection{Mejora de API Collection}
				\subsubsection{Mejora de API Concurrency}
				
			\subsection{Clase optional}
			
			\subsection{forEach e interfaces iterables} 
			Cada vez que necesitamos recorrer una colleci\'on es necesario crear un iterador cuyo prop\'osito es simplemente 
			recorrer la colecci\'on y luego tenemos la l\'ogica en un ciclo por cada uno de los elementos en dicha colecci\'on.
			
			\subsection{M\'etodos default}
			
			\subsection{M\'etodos est\'aticos}
			
			\subsection{Referencias a m\'etodos}
			\newpage
	
		\section{Cambios Java 11}
		\newpage
	
		\section{Cambios Java 17}
		\newpage
			
		\section{Cambios Java 21}
		\newpage
		
		\section{Collections}
			\subsection{Lists}
			\newpage
			\subsection{Sets}
			\newpage
			\subsection{Maps}
		\newpage
		
		\section{Streams}
		\newpage
	
	\chapter{Hibernate}
	\newpage

	\chapter{Spring Boot}
		\section{Anotaciones}
			\subsection{@Repository}
			\subsection{@Service}
			\subsection{@Controller}
			\subsection{@RestController}
				\subsubsection{@GetMapping}
				\subsubsection{@PostMapping}
				\subsubsection{@PutMapping}
	\newpage
		\section{JPA}
	\newpage
			\subsection{Anotaciones}
				\subsubsection{ @Entity }
				\subsubsection{ @Table }
				\subsubsection{ @Id }
				\subsubsection{ @GeneratedValue }
				\subsubsection{ @OneToOne }
				\subsubsection{ @OneToMany }
				\subsubsection{ @ManyToMany }
				\subsubsection{ @JoinColumn }
				\subsubsection{ @JoinTable }
	\newpage
			\subsection{Interfaces Repositorio}
	\newpage
		\section{Pruebas}
			\subsection{Unitarias}
			\subsection{Integraci\'on}
	\newpage
		\section{Seguridad}
			\subsection{OAuth2}
			\subsection{OKTA}
			\subsection{JWT}
	\newpage
			
\part{  Desarrollo Front End}
	\chapter{Angular}
	\newpage
		\section{CLI}
	\newpage
	 	\section{Estructura de una aplicaci\'on}
	\newpage
		\section{Modulos}
	\newpage
		\section{Servicios}
	\newpage
		\section{Componentes}
			\subsection{Compartiendo datos entre componentes}
				\begin{itemize}
					\item Padre a hijo: @Input
					\item Hijo a padre: @Output and EventEmitter
					\item Hijo a padre: @ViewChild
					\item Componentes no relacionados: Usando servicios
				\end{itemize}
	\newpage
	
\end{document}